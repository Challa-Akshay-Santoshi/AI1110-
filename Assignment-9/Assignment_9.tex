%%%%%%%%%%%%%%%%%%%%%%%%%%%%%%%%%%%%%%%%%%%%%%%%%%%%%%%%%%%%%%%
%
% Welcome to Overleaf --- just edit your LaTeX on the left,
% and we'll compile it for you on the right. If you open the
% 'Share' menu, you can invite other users to edit at the same
% time. See www.overleaf.com/learn for more info. Enjoy!
%
%%%%%%%%%%%%%%%%%%%%%%%%%%%%%%%%%%%%%%%%%%%%%%%%%%%%%%%%%%%%%%%

% Inbuilt themes in beamer
\documentclass{beamer}
\usepackage{listings}
\lstset{
%language=C,
frame=single, 
breaklines=true,
columns=fullflexible
}
\usepackage{subcaption}
\usepackage{url}
\usepackage{tikz}
\usepackage{tkz-euclide} % loads  TikZ and tkz-base
%\usetkzobj{all}
\usetikzlibrary{calc,math}
\usepackage{float}
\newcommand\norm[1]{\left\lVert#1\right\rVert}
\renewcommand{\vec}[1]{\mathbf{#1}}
\providecommand{\pr}[1]{\ensuremath{\Pr\left(#1\right)}}
\usepackage[export]{adjustbox}
\usepackage[utf8]{inputenc}
\usepackage{amsmath}


\usepackage{tfrupee}
\usepackage{amsmath}
\usepackage{amssymb}
\usepackage{gensymb}
\usepackage{graphicx}
\usepackage{txfonts}

\def\inputGnumericTable{}

                                
\usepackage{color}                                            
\usepackage{array}                                            
\usepackage{longtable}                                        
\usepackage{calc}                                             
\usepackage{multirow}                                         
\usepackage{hhline}                                           
\usepackage{ifthen}
\usepackage{caption} 
\captionsetup[table]{skip=3pt}  
\providecommand{\pr}[1]{\ensuremath{\Pr\left(#1\right)}}
\providecommand{\cbrak}[1]{\ensuremath{\left\{#1\right\}}}
\renewcommand{\thefigure}{\arabic{table}}
\renewcommand{\thetable}{\arabic{table}}   
\providecommand{\brak}[1]{\ensuremath{\left(#1\right)}}

% Theme choice:
\usetheme{CambridgeUS}


% Title page details: 
\title{Assignment 9} 
\author{Challa Akshay Santoshi - CS21BTECH11012}
\date{\today}
\logo{\large \LaTeX{}}


\begin{document}

% Title page frame
\begin{frame}
    \titlepage 
\end{frame}

% Remove logo from the next slides
\logo{}


% Outline frame
\begin{frame}{Outline}
    \tableofcontents
\end{frame}


% Lists frame
\section{Question}
\begin{frame}{Question}
\begin{block}{CBSE 12 13.5 Q 1} A die is thrown 6 times. If 'getting an odd number' is a success, what is the probability of
\begin{enumerate}
    \item 5 successes ?
    \item at least 5 successes ?
    \item at most 5 successes ?
\end{enumerate}
    \end{block}
\end{frame}



\section{Definitions}
\begin{frame}{Definitions}
Let $X \in \{0,1,2,3,4,5,6\} $ be a random variable denoting the number of successes (getting an odd number) in an experiment of 6 trials.\\
 \begin{table}[ht!]
    \centering
    \input{tables/table}
    \caption{Outcomes of the Experiment}
	\label{table:table1}
\end{table}   
Throwing a die is a Bernoulli trail. So, X has a binomial distribution
\begin{align}
   \pr{X=x} = \frac{n!}{x! (n-x)!}\brak{1-p}^{n-x}p^{x}
\end{align}
\end{frame}

\section{Probability}
\begin{frame}{5 successes}
\begin{align}
    \pr{X=5} &= \frac{6!}{5! (6-5)!}p^5q^{1} \\
    &= 6 \times \brak{\frac{1}{2}}^{5} \times \brak{\frac{1}{2}}^{1}\\
    &= 6 \times \frac{1}{32} \times \frac{1}{2}\\
    &= \frac{3}{32}
\end{align}

\end{frame}
\begin{frame}{at least 5 successes}
\begin{align}
    \pr{X\geq5} &= \pr{X=5} + \pr{X=6}\\
    &= \frac{6!}{5! (6-5)!}p^5q^{1} + \frac{6!}{6! (6-6)!}p^5q^{0}\\
    &= 6 \times \brak{\frac{1}{2}}^{5} \times \brak{\frac{1}{2}}^{1} + 1 \times \brak{\frac{1}{2}}^{6} \times \brak{\frac{1}{2}}^{0}\\
    &= \frac{6}{64} + \frac{1}{64}\\
    &= \frac{7}{64}
\end{align}

\end{frame}
\begin{frame}{at most 5 successes}
\begin{align}
    \pr{X\leq5} &= 1 - \pr{X=6}\\
    &= 1 - \frac{6!}{6! (6-6)!}p^6q^{0} \\
    &= 1 -  1 \times \brak{\frac{1}{2}}^{6} \times \brak{\frac{1}{2}}^{0}\\
    &= 1 -  \frac{1}{64}\\
    &= \frac{63}{64}
\end{align}
\end{frame}

\section{Graph}
\begin{frame}{PMF Graph}
\begin{figure}[!ht]
		\centering
		\includegraphics[width=\textwidth,height=5.5cm,keepaspectratio]{figures/Figure_1.png}
		\caption{Probability Mass Function}
		\label{fig1}
	\end{figure}
\end{frame}

\begin{frame}{CDF Graph}
\begin{figure}[!ht]
		\centering
		\includegraphics[width=\textwidth,height=5.5cm,keepaspectratio]{figures/Figure_2.png}
		\caption{Cummulative Distribution Function}
		\label{fig1}
	\end{figure}
\end{frame}

\end{document}
