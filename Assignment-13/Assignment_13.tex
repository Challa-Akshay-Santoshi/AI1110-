%%%%%%%%%%%%%%%%%%%%%%%%%%%%%%%%%%%%%%%%%%%%%%%%%%%%%%%%%%%%%%%
%
% Welcome to Overleaf --- just edit your LaTeX on the left,
% and we'll compile it for you on the right. If you open the
% 'Share' menu, you can invite other users to edit at the same
% time. See www.overleaf.com/learn for more info. Enjoy!
%
%%%%%%%%%%%%%%%%%%%%%%%%%%%%%%%%%%%%%%%%%%%%%%%%%%%%%%%%%%%%%%%

% Inbuilt themes in beamer
\documentclass{beamer}
\usepackage{listings}
\lstset{
%language=C,
frame=single, 
breaklines=true,
columns=fullflexible
}
\usepackage{subcaption}
\usepackage{url}
\usepackage{tikz}
\usepackage{tkz-euclide} % loads  TikZ and tkz-base
%\usetkzobj{all}
\usetikzlibrary{calc,math}
\usepackage{float}
\newcommand\norm[1]{\left\lVert#1\right\rVert}
\renewcommand{\vec}[1]{\mathbf{#1}}
\providecommand{\pr}[1]{\ensuremath{\Pr\left(#1\right)}}
\usepackage[export]{adjustbox}
\usepackage[utf8]{inputenc}
\usepackage{amsmath}


\usepackage{tfrupee}
\usepackage{amsmath}
\usepackage{amssymb}
\usepackage{gensymb}
\usepackage{graphicx}
\usepackage{txfonts}

\def\inputGnumericTable{}

                                
\usepackage{color}                                            
\usepackage{array}                                            
\usepackage{longtable}                                        
\usepackage{calc}                                             
\usepackage{multirow}                                         
\usepackage{hhline}                                           
\usepackage{ifthen}
\usepackage{caption} 
\captionsetup[table]{skip=3pt}  
\providecommand{\pr}[1]{\ensuremath{\Pr\left(#1\right)}}
\providecommand{\cbrak}[1]{\ensuremath{\left\{#1\right\}}}
\renewcommand{\thefigure}{\arabic{table}}
\renewcommand{\thetable}{\arabic{table}}   
\providecommand{\brak}[1]{\ensuremath{\left(#1\right)}}

% Theme choice:
\usetheme{CambridgeUS}


% Title page details: 
\title{Assignment 13} 
\author{Challa Akshay Santoshi - CS21BTECH11012}
\date{\today}
\logo{\large \LaTeX{}}

\begin{document}

% Title page frame
\begin{frame}
    \titlepage 
\end{frame}

% Remove logo from the next slides
\logo{}


% Outline frame
\begin{frame}{Outline}
    \tableofcontents
\end{frame}


% Lists frame
\section{Question}
\begin{frame}{Question}
\begin{block}{Excercise 11 Quetion 12} Show that, if the process $X(\omega)$ is white noise with zero mean and autocovariance $Q(u)\delta(u-v)$, then its inverse Fourier transform, x(t) is WSS with power spectrum $\frac{Q(\omega)}{2\pi}$.
    \end{block}
\end{frame}

\section{Solution}
\begin{frame}{Definitions}
Inverse Fourier transform of $X(\omega)$ is given by
\begin{align}
    x(t) = \frac{1}{2\pi} \int\limits_{-\infty}^\infty X(\omega) e^{j\omega t} d\omega
\end{align}
From the Fourier-inversion formula, it follows that
\begin{align}
    R(\tau) = \frac{1}{2\pi} \int\limits_{-\infty}^\infty S(\omega) e^{j\omega t} d\omega
\end{align}
The process x(t) is WSS iff $E\{X(\omega)\} = 0$ for $\omega \neq 0$ and 
\begin{align}
    E\{X(u)X^*(v)\} = Q(u)\delta(u-v)
\end{align}
\end{frame}

\begin{frame}{Proof}
Let $x(t_1)$ and $x(t_2)$ be the inverse Fourier transform of X(u) and X(v) respectively.
\begin{align}
    E\{x(t_1)x^*(t_2)\} &= E\{(\frac{1}{2\pi} \int\limits_{-\infty}^\infty X(u)e^{jut_1} du ) (\frac{1}{2\pi} \int\limits_{-\infty}^\infty X^*(v)e^{-jvt_2} dv )\}\\
    &= \frac{1}{4\pi^2} \int\limits_{-\infty}^\infty \int\limits_{-\infty}^\infty E\{X(u)X^*(v)\} e^{j(ut_1 - vt_2)} du dv\\
    &= \frac{1}{4\pi^2} \int\limits_{-\infty}^\infty (\int\limits_{-\infty}^\infty Q(u)\delta(u-v) e^{j(ut_1 - vt_2)} dv) du
\end{align}
\end{frame}

\begin{frame}{Proof}
\begin{align}
    E\{x(t_1)x^*(t_2)\} &= \frac{1}{4\pi^2} \int\limits_{-\infty}^\infty Q(u) e^{ju(t_1 - t_2)} du\\
    &= \frac{1}{4\pi^2} \int\limits_{-\infty}^\infty Q(u) e^{ju\tau} du\\
    &= \frac{1}{2\pi} \int\limits_{-\infty}^\infty \frac{Q(u)}{2\pi} e^{ju \tau} du
\end{align}
\begin{align}
    R_{xx}(\tau) = \frac{1}{2\pi} \int\limits_{-\infty}^\infty \frac{Q(\omega)}{2\pi} e^{j\omega \tau} d\omega
\end{align}
\end{frame}

\begin{frame}{Finding Power Spectrum}
To find the Power Spectrum, we compare the equations given below,
\begin{align}
    R(\tau) = \frac{1}{2\pi} \int\limits_{-\infty}^\infty S(\omega) e^{j\omega t} d\omega\\
    R_{xx}(\tau) = \frac{1}{2\pi} \int\limits_{-\infty}^\infty \frac{Q(\omega)}{2\pi} e^{j\omega \tau} d\omega
\end{align}
Therefore the Power Spectrum is
\begin{align}
    S_{xx} (\omega) = \frac{Q(\omega)}{2\pi}
\end{align}
\end{frame}

\end{document}
