%%%%%%%%%%%%%%%%%%%%%%%%%%%%%%%%%%%%%%%%%%%%%%%%%%%%%%%%%%%%%%%
%
% Welcome to Overleaf --- just edit your LaTeX on the left,
% and we'll compile it for you on the right. If you open the
% 'Share' menu, you can invite other users to edit at the same
% time. See www.overleaf.com/learn for more info. Enjoy!
%
%%%%%%%%%%%%%%%%%%%%%%%%%%%%%%%%%%%%%%%%%%%%%%%%%%%%%%%%%%%%%%%

% Inbuilt themes in beamer
\documentclass{beamer}
\usepackage{listings}
\lstset{
%language=C,
frame=single, 
breaklines=true,
columns=fullflexible
}
\usepackage{subcaption}
\usepackage{url}
\usepackage{tikz}
\usepackage{tkz-euclide} % loads  TikZ and tkz-base
%\usetkzobj{all}
\usetikzlibrary{calc,math}
\usepackage{float}
\newcommand\norm[1]{\left\lVert#1\right\rVert}
\renewcommand{\vec}[1]{\mathbf{#1}}
\providecommand{\pr}[1]{\ensuremath{\Pr\left(#1\right)}}
\usepackage[export]{adjustbox}
\usepackage[utf8]{inputenc}
\usepackage{amsmath}


\usepackage{tfrupee}
\usepackage{amsmath}
\usepackage{amssymb}
\usepackage{gensymb}
\usepackage{graphicx}
\usepackage{txfonts}

\def\inputGnumericTable{}

                                
\usepackage{color}                                            
\usepackage{array}                                            
\usepackage{longtable}                                        
\usepackage{calc}                                             
\usepackage{multirow}                                         
\usepackage{hhline}                                           
\usepackage{ifthen}
\usepackage{caption} 
\captionsetup[table]{skip=3pt}  
\providecommand{\pr}[1]{\ensuremath{\Pr\left(#1\right)}}
\providecommand{\cbrak}[1]{\ensuremath{\left\{#1\right\}}}
\renewcommand{\thefigure}{\arabic{table}}
\renewcommand{\thetable}{\arabic{table}}   
\providecommand{\brak}[1]{\ensuremath{\left(#1\right)}}

% Theme choice:
\usetheme{CambridgeUS}


% Title page details: 
\title{Assignment 11} 
\author{Challa Akshay Santoshi - CS21BTECH11012}
\date{\today}
\logo{\large \LaTeX{}}

\begin{document}

% Title page frame
\begin{frame}
    \titlepage 
\end{frame}

% Remove logo from the next slides
\logo{}


% Outline frame
\begin{frame}{Outline}
    \tableofcontents
\end{frame}


% Lists frame
\section{Question}
\begin{frame}{Question}
\begin{block}{Excercise 7 Quetion 9} Show that if $X_i \geq 0$, $E({X_i}^2) = M$ and $s = \sum_{i=1}^{n}X_i $, then $E(s^2) \leq ME(n^2)$ .
    \end{block}
\end{frame}

\section{Solution}
\begin{frame}{Definitions}
Given a discrete type random variable \textbf{n} taking the values 1,2,... and a sequence of random variables \textbf{$X_k$} independent of \textbf{n}, then the sum \textbf{s} is defined as 
\begin{align}
    s = \sum_{k=1}^{n} X_k
\end{align}
Given that for any k,
\begin{align}
    E({X_k}^2) = M
\end{align}
\end{frame}

\begin{frame}{Proof}
\begin{align}
    (E(X_iX_j))^2 = (E(X_i)E(X_j))^2 \leq E({X_i}^2)E({X_j}^2) = M^2
    \end{align}
    \begin{align}
    (E(X_iX_j))^2 \leq M^2
    \implies E(X_iX_j) \leq M
\end{align}
\begin{align}
    E(s^2|\textbf{n} = n) &= E((\sum_{i=1}^{n}X_i)(\sum_{j=1}^{n}X_j))\\
    &= E(\sum_{i=1}^{n}\sum_{j=1}^{n}X_iX_j)\\
    &= \sum_{i=1}^{n}\sum_{j=1}^{n} E(X_iX_j)\\
    &\leq \sum_{i=1}^{n}\sum_{j=1}^{n} M
\end{align}
\end{frame}

\begin{frame}{Proof}
\begin{align}
    E(s^2|\textbf{n} = n) \leq n^2M
\end{align}
We can write,
\begin{align}
    E(s^2) = E(E(s^2|\textbf{n} = n))\\
    E(s^2|\textbf{n} = n) \leq n^2M\\
    E(E(s^2|\textbf{n} = n)) \leq E(n^2M)\\
    \implies E(s^2) \leq ME(n^2)
\end{align}
Hence proved.
\end{frame}

\end{document}
