\documentclass[journal,12pt,twocolumn]{IEEEtran}

\usepackage{setspace}
\usepackage{gensymb}
\singlespacing
\usepackage[cmex10]{amsmath}

\usepackage{amsthm}

\usepackage{mathrsfs}
\usepackage{txfonts}
\usepackage{stfloats}
\usepackage{bm}
\usepackage{cite}
\usepackage{cases}
\usepackage{subfig}

\usepackage{longtable}
\usepackage{multirow}

\usepackage{enumitem}
\usepackage{mathtools}
\usepackage{steinmetz}
\usepackage{tikz}
\usepackage{circuitikz}
\usepackage{verbatim}
\usepackage{tfrupee}
\usepackage[breaklinks=true]{hyperref}
\usepackage{graphicx}
\usepackage{tkz-euclide}

\usetikzlibrary{calc,math}
\usepackage{listings}
    \usepackage{color}                                            %%
    \usepackage{array}                                            %%
    \usepackage{longtable}                                        %%
    \usepackage{calc}                                             %%
    \usepackage{multirow}                                         %%
    \usepackage{hhline}                                           %%
    \usepackage{ifthen}                                           %%
    \usepackage{lscape}     
\usepackage{multicol}
\usepackage{chngcntr}

\DeclareMathOperator*{\Res}{Res}

\renewcommand\thesection{\arabic{section}}
\renewcommand\thesubsection{\thesection.\arabic{subsection}}
\renewcommand\thesubsubsection{\thesubsection.\arabic{subsubsection}}

\renewcommand\thesectiondis{\arabic{section}}
\renewcommand\thesubsectiondis{\thesectiondis.\arabic{subsection}}
\renewcommand\thesubsubsectiondis{\thesubsectiondis.\arabic{subsubsection}}


\hyphenation{op-tical net-works semi-conduc-tor}
\def\inputGnumericTable{}                                 %%

\lstset{
%language=C,
frame=single, 
breaklines=true,
columns=fullflexible
}
\begin{document}

\newcommand{\BEQA}{\begin{eqnarray}}
\newcommand{\EEQA}{\end{eqnarray}}
\newcommand{\define}{\stackrel{\triangle}{=}}
\bibliographystyle{IEEEtran}
\raggedbottom
\setlength{\parindent}{0pt}
\providecommand{\mbf}{\mathbf}
\providecommand{\pr}[1]{\ensuremath{\Pr\left(#1\right)}}
\providecommand{\qfunc}[1]{\ensuremath{Q\left(#1\right)}}
\providecommand{\sbrak}[1]{\ensuremath{{}\left[#1\right]}}
\providecommand{\lsbrak}[1]{\ensuremath{{}\left[#1\right.}}
\providecommand{\rsbrak}[1]{\ensuremath{{}\left.#1\right]}}
\providecommand{\brak}[1]{\ensuremath{\left(#1\right)}}
\providecommand{\lbrak}[1]{\ensuremath{\left(#1\right.}}
\providecommand{\rbrak}[1]{\ensuremath{\left.#1\right)}}
\providecommand{\cbrak}[1]{\ensuremath{\left\{#1\right\}}}
\providecommand{\lcbrak}[1]{\ensuremath{\left\{#1\right.}}
\providecommand{\rcbrak}[1]{\ensuremath{\left.#1\right\}}}
\theoremstyle{remark}
\newtheorem{rem}{Remark}
\newcommand{\sgn}{\mathop{\mathrm{sgn}}}
\newcommand{\comb}[2]{{}^{#1}\mathrm{C}_{#2}}
\providecommand{\abs}[1]{\vert#1\vert}
\providecommand{\res}[1]{\Res\displaylimits_{#1}} 
\providecommand{\norm}[1]{\lVert#1\rVert}
%\providecommand{\norm}[1]{\lVert#1\rVert}
\providecommand{\mtx}[1]{\mathbf{#1}}
\providecommand{\mean}[1]{E[ #1 ]}
\providecommand{\fourier}{\overset{\mathcal{F}}{ \rightleftharpoons}}
%\providecommand{\hilbert}{\overset{\mathcal{H}}{ \rightleftharpoons}}
\providecommand{\system}{\overset{\mathcal{H}}{ \longleftrightarrow}}
	%\newcommand{\solution}[2]{\textbf{Solution:}{#1}}
\newcommand{\solution}{\noindent \textbf{Solution: }}
\newcommand{\cosec}{\,\text{cosec}\,}
\providecommand{\dec}[2]{\ensuremath{\overset{#1}{\underset{#2}{\gtrless}}}}

\newcommand{\myvec}[1]{\ensuremath{\begin{pmatrix}#1\end{pmatrix}}}
\newcommand{\mydet}[1]{\ensuremath{\begin{vmatrix}#1\end{vmatrix}}}
\numberwithin{equation}{subsection}
\makeatletter
\@addtoreset{figure}{problem}
\makeatother
\let\StandardTheFigure\thefigure
\let\vec\mathbf
\renewcommand{\thefigure}{\theproblem}
\def\putbox#1#2#3{\makebox[0in][l]{\makebox[#1][l]{}\raisebox{\baselineskip}[0in][0in]{\raisebox{#2}[0in][0in]{#3}}}}
     \def\rightbox#1{\makebox[0in][r]{#1}}
     \def\centbox#1{\makebox[0in]{#1}}
     \def\topbox#1{\raisebox{-\baselineskip}[0in][0in]{#1}}
     \def\midbox#1{\raisebox{-0.5\baselineskip}[0in][0in]{#1}}
\vspace{3cm}
\title{Assignment 5}
\author{Challa Akshay Santoshi - CS21BTECH11012}
\maketitle
\newpage
\bigskip
\renewcommand{\thefigure}{\theenumi}
\renewcommand{\thetable}{\theenumi}
\begin{center}
  \textbf{\underline{CBSE Class 11 Probability}}\\
\end{center}
\begin{center}
  \textbf{Excercise: 16.2 Question: 2}  
\end{center}
A dice is thrown. Describe the following events:
\begin{enumerate}
	\item A: a number less than 7
	\item B: a number greater than 7
	\item C: a multiple of 3
	\item D: a number less than 4
	\item E: an even number greater than 4
	\item F: a number not less than 3
\end{enumerate}
Also find $A \cup B $, $A \cap B$, $B \cup C$, $E \cap F$, $D \cap E$, $A - C$, $D - E$, $E \cap \overline{F}$, $\overline{F}$
\begin{center}
  \textbf{Solution:}  
\end{center}

The possible outcomes when dice is thrown:
\begin{align}
    S = \cbrak {1,2,3,4,5,6}
\end{align}
\begin{enumerate}
	\item A: a number less than 7
	     \begin{align}
	         A = \cbrak {1,2,3,4,5,6}
	     \end{align}
	\item B: a number greater than 7
	     \begin{align}
	         B = \phi
	     \end{align}
	\item C: a multiple of 3
	     \begin{align}
	         C = \cbrak {3,6}
	     \end{align}
	\item D: a number less than 4
	     \begin{align}
	         D = \cbrak {1,2,3}
	     \end{align}
	\item E: an even number greater than 4
	     \begin{align}
	         E = \cbrak {6}
	     \end{align}
	\item F: a number not less than 3
	     \begin{align}
	         F = \cbrak {3,4,5,6}
	     \end{align}
\end{enumerate}
Using the events we can compute the following relations:\\
\begin{enumerate}
    \item $A \cup B $
        \begin{align}
            A \cup B &= \cbrak {1,2,3,4,5,6} \cup \phi\\
            &= \cbrak {1,2,3,4,5,6}
        \end{align}
    \item $A \cap B $
        \begin{align}
            A \cap B &= \cbrak {1,2,3,4,5,6} \cap \phi\\
            &= \phi
        \end{align}
    \item $B \cup C $
        \begin{align}
            B \cup C &= \phi \cup \cbrak {3,6}\\
            &= \cbrak {3,6}
        \end{align}
    \item $E \cap F $
        \begin{align}
            E \cap F &= \cbrak {6} \cap \cbrak {3,4,5,6}\\
            &= \cbrak {6}
        \end{align}
    \item $D \cap E $
        \begin{align}
            D \cap E &= \cbrak {1,2,3} \cap \cbrak {6}\\
            &= \phi
        \end{align}
    \item $A - C $
        \begin{align}
            A - C &= \cbrak {1,2,3,4,5,6} - \cbrak {3,6}\\
            &= \cbrak {1,2,4,5}
        \end{align}
    \item $D - E $
        \begin{align}
            D - E &= \cbrak {1,2,3} - \cbrak {6}\\
            &= \cbrak {1,2,3}
        \end{align}
    \item $E \cap \overline{F}$
        \begin{align}
            E \cap \overline{F} &= \cbrak {6} \cap \cbrak{1,2}\\
            &= \phi
        \end{align}
    \item $\overline{F}$
        \begin{align}
            \overline{F} &= \cbrak {1,2}
        \end{align}
\end{enumerate}

\end{document}