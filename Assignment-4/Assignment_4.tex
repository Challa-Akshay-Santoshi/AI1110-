\documentclass[journal,12pt,twocolumn]{IEEEtran}

\usepackage{setspace}
\usepackage{gensymb}
\singlespacing
\usepackage[cmex10]{amsmath}

\usepackage{amsthm}

\usepackage{mathrsfs}
\usepackage{txfonts}
\usepackage{stfloats}
\usepackage{bm}
\usepackage{cite}
\usepackage{cases}
\usepackage{subfig}

\usepackage{longtable}
\usepackage{multirow}

\usepackage{enumitem}
\usepackage{mathtools}
\usepackage{steinmetz}
\usepackage{tikz}
\usepackage{circuitikz}
\usepackage{verbatim}
\usepackage{tfrupee}
\usepackage[breaklinks=true]{hyperref}
\usepackage{graphicx}
\usepackage{tkz-euclide}

\usetikzlibrary{calc,math}
\usepackage{listings}
    \usepackage{color}                                            %%
    \usepackage{array}                                            %%
    \usepackage{longtable}                                        %%
    \usepackage{calc}                                             %%
    \usepackage{multirow}                                         %%
    \usepackage{hhline}                                           %%
    \usepackage{ifthen}                                           %%
    \usepackage{lscape}     
\usepackage{multicol}
\usepackage{chngcntr}

\DeclareMathOperator*{\Res}{Res}

\renewcommand\thesection{\arabic{section}}
\renewcommand\thesubsection{\thesection.\arabic{subsection}}
\renewcommand\thesubsubsection{\thesubsection.\arabic{subsubsection}}

\renewcommand\thesectiondis{\arabic{section}}
\renewcommand\thesubsectiondis{\thesectiondis.\arabic{subsection}}
\renewcommand\thesubsubsectiondis{\thesubsectiondis.\arabic{subsubsection}}


\hyphenation{op-tical net-works semi-conduc-tor}
\def\inputGnumericTable{}                                 %%

\lstset{
%language=C,
frame=single, 
breaklines=true,
columns=fullflexible
}
\begin{document}

\newcommand{\BEQA}{\begin{eqnarray}}
\newcommand{\EEQA}{\end{eqnarray}}
\newcommand{\define}{\stackrel{\triangle}{=}}
\bibliographystyle{IEEEtran}
\raggedbottom
\setlength{\parindent}{0pt}
\providecommand{\mbf}{\mathbf}
\providecommand{\pr}[1]{\ensuremath{\Pr\left(#1\right)}}
\providecommand{\qfunc}[1]{\ensuremath{Q\left(#1\right)}}
\providecommand{\sbrak}[1]{\ensuremath{{}\left[#1\right]}}
\providecommand{\lsbrak}[1]{\ensuremath{{}\left[#1\right.}}
\providecommand{\rsbrak}[1]{\ensuremath{{}\left.#1\right]}}
\providecommand{\brak}[1]{\ensuremath{\left(#1\right)}}
\providecommand{\lbrak}[1]{\ensuremath{\left(#1\right.}}
\providecommand{\rbrak}[1]{\ensuremath{\left.#1\right)}}
\providecommand{\cbrak}[1]{\ensuremath{\left\{#1\right\}}}
\providecommand{\lcbrak}[1]{\ensuremath{\left\{#1\right.}}
\providecommand{\rcbrak}[1]{\ensuremath{\left.#1\right\}}}
\theoremstyle{remark}
\newtheorem{rem}{Remark}
\newcommand{\sgn}{\mathop{\mathrm{sgn}}}
\newcommand{\comb}[2]{{}^{#1}\mathrm{C}_{#2}}
\providecommand{\abs}[1]{\vert#1\vert}
\providecommand{\res}[1]{\Res\displaylimits_{#1}} 
\providecommand{\norm}[1]{\lVert#1\rVert}
%\providecommand{\norm}[1]{\lVert#1\rVert}
\providecommand{\mtx}[1]{\mathbf{#1}}
\providecommand{\mean}[1]{E[ #1 ]}
\providecommand{\fourier}{\overset{\mathcal{F}}{ \rightleftharpoons}}
%\providecommand{\hilbert}{\overset{\mathcal{H}}{ \rightleftharpoons}}
\providecommand{\system}{\overset{\mathcal{H}}{ \longleftrightarrow}}
	%\newcommand{\solution}[2]{\textbf{Solution:}{#1}}
\newcommand{\solution}{\noindent \textbf{Solution: }}
\newcommand{\cosec}{\,\text{cosec}\,}
\providecommand{\dec}[2]{\ensuremath{\overset{#1}{\underset{#2}{\gtrless}}}}

\newcommand{\myvec}[1]{\ensuremath{\begin{pmatrix}#1\end{pmatrix}}}
\newcommand{\mydet}[1]{\ensuremath{\begin{vmatrix}#1\end{vmatrix}}}
\numberwithin{equation}{subsection}
\makeatletter
\@addtoreset{figure}{problem}
\makeatother
\let\StandardTheFigure\thefigure
\let\vec\mathbf
\renewcommand{\thefigure}{\theproblem}
\def\putbox#1#2#3{\makebox[0in][l]{\makebox[#1][l]{}\raisebox{\baselineskip}[0in][0in]{\raisebox{#2}[0in][0in]{#3}}}}
     \def\rightbox#1{\makebox[0in][r]{#1}}
     \def\centbox#1{\makebox[0in]{#1}}
     \def\topbox#1{\raisebox{-\baselineskip}[0in][0in]{#1}}
     \def\midbox#1{\raisebox{-0.5\baselineskip}[0in][0in]{#1}}
\vspace{3cm}
\title{Assignment 4}
\author{Challa Akshay Santoshi-CS21BTECH11012}
\maketitle
\newpage
\bigskip
\renewcommand{\thefigure}{\theenumi}
\renewcommand{\thetable}{\theenumi}
\begin{center}
  \textbf{\underline{CBSE Class 10 Probability}}\\
\end{center}
\begin{center}
  \textbf{Excercise: 15.1 Question: 14}  
\end{center}
One card is drawn from a well-shuffled deck of 52 cards. Find the probability of getting
\begin{enumerate}
	\item a king of red colour
	\item a face card
	\item a red face card
	\item the jack of hearts
	\item a spade
	\item the queen of diamonds
\end{enumerate}
\begin{center}
  \textbf{Solution:}  
\end{center}

Let $X \in \{0,1,2,3\} $ be a random variable representing different suits in a deck of cards, that is, clubs, diamonds, hearts and spades.\\
Let $Y \in \{0,1,2\} $ be a random variable representing the face cards, that is, King, Queen and Jack.
\begin{table}[ht!]
\input{tables/tables1}
\caption{Probable Events Representation}
\label{tables:table1}
\end{table}\\
Diamonds and Spades are red coloured cards.\\
Clubs and Hearts are black coloured cards.\\
King, Queen and Jack are the face cards.\\
\begin{enumerate}
    \item The probability of getting a king of red colour
        \begin{multline}
			\pr{Y=0|X=1} + \pr{Y=0|X=3} \\
	   	= \frac{1}{52} + \frac{1}{52}
        = \frac{1}{26}
			\end{multline}
	\item The probability of getting a face card 
	   \begin{multline}
	   	\pr{Y=0} + \pr{Y=1} + \pr{Y=2} \\
	   	= \frac{1}{13} + \frac{1}{13} + \frac{1}{13}
        = \frac{3}{13}
	   \end{multline}
	\item The probability of getting a red face card
	   \begin{multline}
	   	\pr{Y=0|X=1} + \pr{Y=0|X=3} + \\\pr{Y=1|X=1} + \pr{Y=1|X=3} +\\ \pr{Y=2|X=1} + \pr{Y=2|X=3}\\ 
	   	= \frac{1}{52} + \frac{1}{52} + \frac{1}{52} + \frac{1}{52} + \frac{1}{52} + \frac{1}{52} 
        = \frac{3}{26}
	   \end{multline}
	\item The probability of getting the jack of hearts
	   \begin{align}
	   	\pr{Y=2|X=2} 
	   	= \frac{1}{52} 
	   \end{align}
	 \item The probability of getting a spade 
	   \begin{align}
	   	\pr{X=3} 
	   	= \frac{1}{4} 
	   \end{align} 
	  \item The probability of getting the queen of diamonds 
	   \begin{align}
	   	\pr{Y=1|X=1} 
	   	= \frac{1}{52} 
	   \end{align} 
\end{enumerate}

\end{document}