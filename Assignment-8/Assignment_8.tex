%%%%%%%%%%%%%%%%%%%%%%%%%%%%%%%%%%%%%%%%%%%%%%%%%%%%%%%%%%%%%%%
%
% Welcome to Overleaf --- just edit your LaTeX on the left,
% and we'll compile it for you on the right. If you open the
% 'Share' menu, you can invite other users to edit at the same
% time. See www.overleaf.com/learn for more info. Enjoy!
%
%%%%%%%%%%%%%%%%%%%%%%%%%%%%%%%%%%%%%%%%%%%%%%%%%%%%%%%%%%%%%%%

% Inbuilt themes in beamer
\documentclass{beamer}
\usepackage{listings}
\lstset{
%language=C,
frame=single, 
breaklines=true,
columns=fullflexible
}
\usepackage{subcaption}
\usepackage{url}
\usepackage{tikz}
\usepackage{tkz-euclide} % loads  TikZ and tkz-base
%\usetkzobj{all}
\usetikzlibrary{calc,math}
\usepackage{float}
\newcommand\norm[1]{\left\lVert#1\right\rVert}
\renewcommand{\vec}[1]{\mathbf{#1}}
\providecommand{\pr}[1]{\ensuremath{\Pr\left(#1\right)}}
\usepackage[export]{adjustbox}
\usepackage[utf8]{inputenc}
\usepackage{amsmath}


\usepackage{tfrupee}
\usepackage{amsmath}
\usepackage{amssymb}
\usepackage{gensymb}
\usepackage{graphicx}
\usepackage{txfonts}

\def\inputGnumericTable{}

                                
\usepackage{color}                                            
\usepackage{array}                                            
\usepackage{longtable}                                        
\usepackage{calc}                                             
\usepackage{multirow}                                         
\usepackage{hhline}                                           
\usepackage{ifthen}
\usepackage{caption} 
\captionsetup[table]{skip=3pt}  
\providecommand{\pr}[1]{\ensuremath{\Pr\left(#1\right)}}
\providecommand{\cbrak}[1]{\ensuremath{\left\{#1\right\}}}
\renewcommand{\thefigure}{\arabic{table}}
\renewcommand{\thetable}{\arabic{table}}   
\providecommand{\brak}[1]{\ensuremath{\left(#1\right)}}

% Theme choice:
\usetheme{CambridgeUS}


% Title page details: 
\title{Assignment 8} 
\author{Challa Akshay Santoshi - CS21BTECH11012}
\date{\today}
\logo{\large \LaTeX{}}

\begin{document}

% Title page frame
\begin{frame}
    \titlepage 
\end{frame}

% Remove logo from the next slides
\logo{}


% Outline frame
\begin{frame}{Outline}
    \tableofcontents
\end{frame}


% Lists frame
\section{Question}
\begin{frame}{Question}
\begin{block}{CBSE 12 13.4 Q 13} Let X denote the sum of the numbers obtained when two fair dice are rolled. Find the variance and standard deviation of X.
    \end{block}
\end{frame}

\section{Definitions}
\begin{frame}{Definitions}
Let X be a random variable representing the sum of numbers
   \begin{table}[ht!]
    \centering
    \input{tables/Table1}
    \caption{Random Variable $X$}
	\label{table:table1}
\end{table} 
\end{frame}

\section{Mean}
\begin{frame}{Mean of X}
\begin{align}
    E(X) &= \sum_{i=2}^{12} i\times \pr{X=i}
\end{align}

\begin{multline}
    E(X) = 2 \times \frac{1}{36} + 3 \times \frac{2}{36} + 4 \times \frac{3}{36} + 5 \times \frac{4}{36} + 6 \times \frac{5}{36} + 7 \times \frac{6}{36}\\ + 8 \times \frac{5}{36} + 9 \times \frac{4}{36} + 10 \times \frac{3}{36} + 11 \times \frac{2}{36} + 12\times \frac{1}{36}
\end{multline}
\begin{align}
    E(X) &= \frac{2}{36} + \frac{6}{36} + \frac{12}{36} + \frac{20}{36} + \frac{30}{36} + \frac{42}{36} + \frac{40}{36} + \frac{36}{36} + \frac{30}{36} + \frac{22}{36} + \frac{12}{36}\\
    &= \frac{252}{36}\\
    &= 7
\end{align}
    

   
\end{frame}

\section{Variance}
\begin{frame}{Variance}
\begin{align}
    Var &= E(X^2) - (E(X))^2\\
    &= \sum_{i=2}^{12} i^2\times \pr{X=i} - (\sum_{i=2}^{12} i\times \pr{X=i})^2
\end{align}
Calculation:
\begin{multline}
    Var = 2^2 \times \frac{1}{36} + 3^2 \times \frac{2}{36} + 4^2 \times \frac{3}{36} + 5^2 \times \frac{4}{36} + 6^2 \times \frac{5}{36}  + 7^2 \times \frac{6}{36} \\+ 8^2 \times \frac{5}{36} + 9^2 \times \frac{4}{36} + 10^2 \times \frac{3}{36} + 11^2 \times \frac{2}{36} + 12^2 \times \frac{1}{36} - 7^2
\end{multline}
\begin{multline}
    Var = \frac{4}{36} + \frac{18}{36} + \frac{48}{36} + \frac{100}{36} + \frac{180}{36} + \frac{294}{36} + \frac{320}{36} + \frac{324}{36} + \frac{300}{36} + \frac{242}{36} + \frac{144}{36} - 7^2
\end{multline}

    

\end{frame}

\section{Standard Deviation}
\begin{frame}{Standard Deviation}
\begin{align}
   Var &= \frac{1974}{36} - 49\\
    &= \frac{35}{6}
\end{align}
To calculate Standard Deviation:
\begin{align}
    S.D &= \sqrt{Var}\\
    &= \sqrt{\frac{35}{6}}\\
    &= 2.415
\end{align}
\end{frame}


\end{document}
