%%%%%%%%%%%%%%%%%%%%%%%%%%%%%%%%%%%%%%%%%%%%%%%%%%%%%%%%%%%%%%%
%
% Welcome to Overleaf --- just edit your LaTeX on the left,
% and we'll compile it for you on the right. If you open the
% 'Share' menu, you can invite other users to edit at the same
% time. See www.overleaf.com/learn for more info. Enjoy!
%
%%%%%%%%%%%%%%%%%%%%%%%%%%%%%%%%%%%%%%%%%%%%%%%%%%%%%%%%%%%%%%%

% Inbuilt themes in beamer
\documentclass{beamer}

%packages:
% \usepackage{tfrupee}
% \usepackage{amsmath}
% \usepackage{amssymb}
% \usepackage{gensymb}
% \usepackage{txfonts}

% \def\inputGnumericTable{}

% \usepackage[latin1]{inputenc}                                 
% \usepackage{color}                                            
% \usepackage{array}                                            
% \usepackage{longtable}                                        
% \usepackage{calc}                                             
% \usepackage{multirow}                                         
% \usepackage{hhline}                                           
% \usepackage{ifthen}
% \usepackage{caption} 
% \captionsetup[table]{skip=3pt}  
% \providecommand{\pr}[1]{\ensuremath{\Pr\left(#1\right)}}
% \providecommand{\cbrak}[1]{\ensuremath{\left\{#1\right\}}}
% %\renewcommand{\thefigure}{\arabic{table}}
% \renewcommand{\thetable}{\arabic{table}}      

\setbeamertemplate{caption}[numbered]{}

\usepackage{enumitem}
\usepackage{tfrupee}
\usepackage{amsmath}
\usepackage{amssymb}
\usepackage{gensymb}
\usepackage{graphicx}
\usepackage{txfonts}

\def\inputGnumericTable{}

\usepackage[latin1]{inputenc}                                 
\usepackage{color}                                            
\usepackage{array}                                            
\usepackage{longtable}                                        
\usepackage{calc}                                             
\usepackage{multirow}                                         
\usepackage{hhline}                                           
\usepackage{ifthen}
\usepackage{caption} 
\captionsetup[table]{skip=3pt}  
\providecommand{\pr}[1]{\ensuremath{\Pr\left(#1\right)}}
\providecommand{\cbrak}[1]{\ensuremath{\left\{#1\right\}}}
\renewcommand{\thefigure}{\arabic{table}}
\renewcommand{\thetable}{\arabic{table}}   
\providecommand{\brak}[1]{\ensuremath{\left(#1\right)}}

% Theme choice:
\usetheme{CambridgeUS}


% Title page details: 
\title{Assignment 8} 
\author{Challa Akshay Santoshi - CS21BTECH11012}
\date{\today}
\logo{\large \LaTeX{}}


\begin{document}

% Title page frame
\begin{frame}
    \titlepage 
\end{frame}

% Remove logo from the next slides
\logo{}


% Outline frame
\begin{frame}{Outline}
    \tableofcontents
\end{frame}


% Lists frame
\section{Question}
\begin{frame}{Question}
\begin{block}{CBSE 12 13.4 Q 13} Let X denote the sum of the numbers obtained when two fair dice are rolled. Find the variance and standard deviation of X.
    \end{block}
\end{frame}

\section{Definitions}
\begin{frame}{Definitions}
Let X be a random variable representing the sum of numbers
   \begin{table}[ht!]
    \centering
    \input{tables/Table1}
    \caption{Random Variable $X$}
	\label{table:table1}
\end{table} 
\end{frame}

\section{Mean}
\begin{frame}{Mean of X}
\begin{align}
    E(X) &= \sum_{i=2}^{12} i\times \pr{X=i}\\
    &= \frac{2+6+12+20+30+42+40+36+30+22+12}{36}\\
    &= \frac{252}{36}\\
    &= 7
\end{align}
\end{frame}

\section{Variance}
\begin{frame}{Variance}
\begin{align}
    Var &= E(X^2) - (E(X))^2\\
    &= \sum_{i=2}^{12} i^2\times \pr{X=i} - (\sum_{i=2}^{12} i\times \pr{X=i})^2\\
    &= \frac{4+18+48+100+180+294+320+324+300+242}{36} - 7^2\\
    &= \frac{1974}{36} - 49\\
    &= \frac{35}{6}
\end{align}
\end{frame}

\section{Standard Deviation}
\begin{frame}{Standard Deviation}
\begin{align}
    S.D &= \sqrt{Var}\\
    &= \sqrt{\frac{35}{6}}\\
    &= 2.415
\end{align}
\end{frame}


\end{document}
